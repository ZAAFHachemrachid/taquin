\documentclass[a4paper]{article}
\usepackage{listings}
% Define Rust language for listings
\lstdefinelanguage{Rust}{
  keywords={
    self,mut,fn,pub,struct,enum,impl,for,match,if,else,let,return,
    where,unsafe,trait,pub,use,mod,as,break,box,continue,crate,move,
    ref,in,type,while,yield,async,await,dyn,abstract,become,final,
    macro,override,priv,try,typeof,unsized,virtual,loop,const,static
  },
  sensitive=true,
  morecomment=[l]{//},
  morecomment=[s]{/*}{*/},
  morestring=[b]",
  morestring=[b]',
  morekeywords={u8,u16,u32,u64,i8,i16,i32,i64,bool,Vec,String,Option,Some,None}
}
\usepackage{color}
\usepackage{graphicx}
\usepackage{amsmath}
\usepackage{algorithm}
\usepackage{algpseudocode}
\usepackage{hyperref}
\usepackage[a4paper,width=6.5in,top=1in,bottom=1in]{geometry}
\usepackage{fancyhdr}
\usepackage{titlesec}

% Enable better line breaking
\sloppy
\emergencystretch=1em

% Define header and footer
\pagestyle{fancy}
\fancyhf{}
\fancyhead[L]{\textbf{Taquin Puzzle Solver Documentation}}
\fancyfoot[C]{\thepage}

% Style section titles
\titleformat{\section}
  {\normalfont\Large\bfseries}{\thesection}{1em}{}
\titleformat{\subsection}
  {\normalfont\Large\itshape}{\thesubsection}{1em}{}

\definecolor{codegreen}{rgb}{0,0.6,0}
\definecolor{codegray}{rgb}{0.5,0.5,0.5}
\definecolor{codepurple}{rgb}{0.58,0,0.82}
\definecolor{backcolour}{rgb}{0.95,0.95,0.92}

\lstdefinestyle{rustcode}{
    backgroundcolor=\color{backcolour},
    commentstyle=\color{codegreen},
    keywordstyle=\color{blue},
    numberstyle=\tiny\color{codegray},
    stringstyle=\color{codepurple},
    basicstyle=\ttfamily\footnotesize,
    breakatwhitespace=false,
    breaklines=true,
    captionpos=b,
    keepspaces=true,
    numbers=left,
    numbersep=5pt,
    showspaces=false,
    showstringspaces=false,
    showtabs=false,
    tabsize=2,
    language=Rust
}

\title{Taquin Puzzle Solver Implementation}
\author{Technical Documentation}
\date{\today}

\begin{document}

\maketitle

\tableofcontents

\section{Introduction}
This document describes the implementation of a Taquin (8-puzzle) solver in Rust. The project implements three different search algorithms to solve the puzzle: A* Search, Breadth-First Search (BFS), and Depth-First Search (DFS). Each algorithm offers different characteristics in terms of solution optimality and resource usage.

\section{Project Structure}
The project is organized into several modules:
\begin{itemize}
    \item \texttt{common/}: Contains the core game logic and board representation
    \item \texttt{A\_star/}: Implementation of the A* search algorithm
    \item \texttt{BFS/}: Implementation of the Breadth-First Search
    \item \texttt{DFS/}: Implementation of the Depth-First Search
\end{itemize}

\section{Board Implementation}
\subsection{Data Structure}
The board is represented by the \texttt{Board} struct and includes a direction enum:

\begin{lstlisting}[style=rustcode]
#[derive(Clone, Debug, PartialEq, Eq)]
pub enum Direction {
    Up,
    Down,
    Left,
    Right,
}

#[derive(Clone, Debug, PartialEq, Eq, Hash)]
pub struct Board {
    state: Vec<u8>,
    blank_pos: usize,
    size: usize,
}
\end{lstlisting}

The implementation includes:
\begin{itemize}
    \item A 1D vector representing the puzzle state for better performance
    \item Tracking of the blank tile position using a single index
    \item Support for different board sizes (optimized for 3x3)
    \item Efficient index-based calculations for moves and state transitions
\end{itemize}

\subsection{Core Functionality}
Key methods include:
\begin{itemize}
    \item \texttt{new()}: Creates a new board from a 2D initial state
    \item \texttt{is\_goal()}: Checks if the current state matches the goal state
    \item \texttt{make\_move()}: Executes a move in a given direction
    \item \texttt{get\_possible\_moves()}: Returns valid moves from the current state
    \item \texttt{manhattan\_distance()}: Calculates heuristic for A* search
\end{itemize}

\section{Search Algorithms}
\subsection{A* Search}
The A* implementation uses a sophisticated State struct with priority queue ordering:

\begin{lstlisting}[style=rustcode]
#[derive(Clone, Eq)]
struct State {
    state: Vec<u8>,
    blank_pos: usize,
    size: usize,
    path: Vec<Direction>,
    g_cost: u32, // Cost from start to current node
    h_cost: u32, // Heuristic cost (Manhattan distance)
}
\end{lstlisting}

Key components:
\begin{itemize}
    \item Efficient state representation using 1D vector
    \item Path tracking for move reconstruction
    \item g\_cost tracking actual path cost
    \item h\_cost using Manhattan distance heuristic
    \item Custom ordering implementation for optimal state exploration
\end{itemize}

The solver implements the Solver trait:
\begin{lstlisting}[style=rustcode]
impl Solver for AStarSolver {
    fn solve(&self, _optimal_length: Option<usize>) -> Option<SolutionInfo>
}
\end{lstlisting}

\subsection{Breadth-First Search}
BFS implementation uses a queue to explore states level by level:

\begin{algorithm}
\caption{BFS Implementation}
\begin{algorithmic}[1]
\State Initialize queue with start state
\State Initialize visited set
\While{queue not empty}
    \State current\_state = queue.pop\_front()
    \If{current\_state is goal}
        \Return solution path
    \EndIf
    \For{each possible move}
        \State new\_state = apply move to current\_state
        \If{new\_state not visited}
            \State Add new\_state to queue
            \State Mark new\_state as visited
        \EndIf
    \EndFor
\EndWhile
\Return no solution
\end{algorithmic}
\end{algorithm}

Key features:
\begin{itemize}
    \item Guarantees shortest path solution
    \item Memory intensive but complete
    \item Compares solution length with A* optimal length
\end{itemize}

\subsection{Depth-First Search}
DFS implementation uses recursive exploration with iterative deepening:

\begin{algorithm}
\caption{DFS Implementation}
\begin{algorithmic}[1]
\For{depth\_limit = 1 to 20}
    \State Clear visited set
    \State result = DFS\_recursive(start\_state, depth\_limit)
    \If{result is solution}
        \Return solution path
    \EndIf
\EndFor
\Return no solution
\end{algorithmic}
\end{algorithm}

Key features:
\begin{itemize}
    \item Uses iterative deepening to prevent infinite paths
    \item Memory efficient but may not find optimal solution
    \item Implements backtracking to explore all possibilities
\end{itemize}

\section{Algorithm Comparison}

\begin{tabular}{|l|l|l|l|}
\hline
\textbf{Algorithm} & \textbf{Optimality} & \textbf{Memory Usage} & \textbf{Speed} \\
\hline
A* & Optimal & Medium & Fast \\
BFS & Optimal & High & Medium \\
DFS & Not Optimal & Low & Variable \\
\hline
\end{tabular}

\section{Testing and Validation}
The implementation includes comprehensive test coverage:
\begin{itemize}
    \item Unit tests for board operations
    \item Integration tests for each solver
    \item Tests for different puzzle sizes (2x2, 3x3)
    \item Edge case testing (already solved puzzles)
    \item Solution validity verification
\end{itemize}

Example test cases:
\begin{lstlisting}[style=rustcode]
#[test]
fn test_solve_simple_puzzle() {
    let initial_state = vec![vec![1, 0], vec![3, 2]];
    let board = Board::new(initial_state);
    let solver = AStarSolver::new(board.clone());
    let solution = solver.solve(None);
    assert!(solution.is_some(), "Should find a solution");
}

#[test]
fn test_solution_validity() {
    let initial_state = vec![vec![1, 0], vec![2, 3]];
    let board = Board::new(initial_state);
    let solver = AStarSolver::new(board.clone());
    let solution = solver.solve(None).unwrap();
    
    // Verify solution reaches goal state
    let mut test_board = board;
    for move_dir in solution.moves {
        assert!(test_board.make_move(move_dir).is_ok());
    }
    assert!(test_board.is_goal());
}
\end{lstlisting}

\section{Display Implementation}
The Board struct implements the Display trait for visualization:
\begin{itemize}
    \item ASCII box drawing characters for borders
    \item Clear representation of the blank tile
    \item Aligned grid format for readability
\end{itemize}

Example output:
\begin{verbatim}
+---+---+---+
| 1 | 2 | 3 |
+---+---+---+
| 4 | _ | 6 |
+---+---+---+
| 7 | 5 | 8 |
+---+---+---+
\end{verbatim}

\section{Conclusion}
The implementation provides a comprehensive solution to the Taquin puzzle problem, offering three different search strategies. The A* algorithm provides the most efficient balance between optimality and performance, while BFS and DFS offer alternative approaches with different trade-offs in terms of memory usage and solution quality. The codebase emphasizes performance through efficient data structures, comprehensive testing, and clear visualization capabilities.

\end{document}
